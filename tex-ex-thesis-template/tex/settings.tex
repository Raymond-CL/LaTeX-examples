% settings for ctexart document, only uncomment options that you know
\documentclass[
	UTF8,
%	zhmap=true,	%true,false,zhmCJK
%	fontset=none,	%adobe,fandol,founder,mac,macnew,macold,ubuntu,windows,none
%	\songti\heiti\fangsong\kaishu\lishu\youyuan\yahei\pingfang
	zihao=5,	%-4,5,false
%	10pt,	%10pt,11pt,12pt, this supercedes zihao font size
%	heading=true,	%true,false
%	sub3section,	%sub3section,sub4section
%	scheme=chinese,	%chinese,plain
%	linespread=1.3,	%1.3,1.4
]{ctexart}

% included packages, comment out useless ones to speed-up compilation
\usepackage{geometry}
\usepackage{amsmath,amsfonts,amssymb,amsthm}
\usepackage{graphicx}
\usepackage[en-US]{datetime2}
\usepackage[strict]{changepage}
%\usepackage{lipsum}
\usepackage{fancyhdr}
%\usepackage{afterpage}
\usepackage{tocloft}
\usepackage{hyperref}		%hyper reference for figs, links, refs, eqs
\usepackage[all]{hypcap}		%hyperref to top of figs.
\usepackage{slashed}
\usepackage{array}
\usepackage{booktabs}
%\usepackage{xcolor}
%\usepackage{setspace}
%\usepackage{MnSymbol}
%\usepackage{epsfig}
%\usepackage{natbib}
%\usepackage{multicol}
%\usepackage{extarrows}
%\usepackage{booktabs}
%\usepackage{color}
%\usepackage[dvipsnames]{xcolor}
%\usepackage{xeCJK,fontspec}
\usepackage{cite}

% settings for ctexart class, only uncomment options that you know
\ctexset{
%	punct=plain,	%quanjiao,banjiao,kaiming,CCT,plain
%	space=auto,	%true,false,auto
%	autoindent=false,	%true,false,#
%	linestretch=\ccwd,	%\ccwd,\maxdimen
	today=old,	%small,big,old
	contentsname={目\hspace{2\ccwd}录},
%	listfigurename={图列测试},
%	listtablename={表列测试},
%	figurename={图测试},
%	tablename={表测试},
%	abstractname={摘要测试},
%	indexname={测试},
	appendixname={附录},
	bibname={参考文献},
%	proofname={测试},
%	refname={测试},
%	algorithmname={测试},
%	continuation={测试},
%	secnumdepth=1,
	section={
		name={第,章},	%{\S},{第,节},{\sectionname\space},{}
%		numbering=true,	%true,false
%		number={\chinese{section}},	%\arabic,\chinese,\roman,\Roman
%		format+={\bfseries\heiti\zihao{3}},
%		nameformat+={},
%		numberformat+={},
%		titleformat+={},
%		aftername+={},
%		aftertitle+={},
%		runin=false,	%true,false
%		hang=true,	%true,false
%		indent=0pt,	%em,ex,\ccwd,pt
%		beforeskip={0pt},afterskip={0pt},
%		fixskip=false,	%true,false
%		afterindent=false,	%true,false
	},
%	subsection={
%		format={\bfseries\heiti\zihao{4}},
%	},
%	tocdepth=1,
}

% settings for geometry package, set before writing in geometry package
\geometry{
	a4paper,
	portrait,
	twoside,
	top=40mm,
	bottom=35mm,
	inner=30mm,
	outer=30mm,
	headheight=25mm,
	headsep=5mm,
	footskip=25mm,
	marginparsep=5mm,
	marginparwidth=15mm,
}

% settings for header and foot style in fancyhdr package
%\renewcommand{\headrulewidth}{0.4pt}
%\renewcommand{\footrulewidth}{0pt}
\fancyhf{}
%\fancyhead[LO]{\includegraphics[height=15mm]{fig-logos/header-doctorate}}
%\fancyhead[RO,LE]{\leftmark}
%\fancyhead[RE]{\rightmark}
%\fancyfoot[C]{-~\thepage~-}
%\fancypagestyle{plain}{
%	\fancyhf{}
%	\fancyhead[LO]{\includegraphics[height=15mm]{fig-logos/header-doctorate}}
%	\fancyhead[RO,LE]{\leftmark}
%	\fancyfoot[C]{-~\thepage~-}
%}
%\fancyhead[LO]{\includegraphics[height=15mm]{fig-logos/header-doctorate}}
%\fancyhead[RO]{\thepage}
\fancyhead[RO]{\leftmark}
\fancyhead[LE]{\rightmark}
\fancyfoot[RO,LE]{-~\thepage~-}
%\if{false}
\fancypagestyle{plain}{
	\fancyhf{}
%	\fancyhead[LO]{\includegraphics[height=15mm]{fig-logos/header-doctorate}}
%	\fancyhead[RO]{\thepage}
%	\fancyhead[LE]{\leftmark}
%	\fancyfoot[CE]{-~\thepage~-}
}
%\fi
\pagestyle{fancy}

% settings for table-of-contents style (dots) in tocloft package
\newcommand{\intotoc}[2][section]{%
    \markboth{\MakeUppercase{#2}}{}% set the leftmark
    \phantomsection% use for adobe bookmark.
    \addcontentsline{toc}{#1}{#2}% add content #2 to toc as #1
}
\renewcommand{\cftsecleader}{\cftdotfill{\cftdotsep}}

% user defined shortcuts
\def\boxtext[#1]#2{\hbox to #1{#2}}
\def\boxuline[#1]#2{\underline{\hbox to #1{\hfill{#2}\hfill}}}
\def\datecn{\the\year~年~\ifnum\the\month <10 0\fi\the\month~月}
\def\dateen{	\DTMenglishmonthname{\the\month}~\the\year}

\newcommand{\blankpage}{
	\pagebreak
	\thispagestyle{empty}
	\vspace*{\stretch{1}}
	\begin{center}
		\zihao{-5}\songti
		此页故意留白\\
		This page intentionally left blank
	\end{center}
	\vspace*{\stretch{2}}
	\pagebreak
}

\numberwithin{equation}{section}
\numberwithin{figure}{section}
\numberwithin{table}{section}

\newcolumntype{L}{>{$\displaystyle }l<{$}}
\newcolumntype{C}{>{$\displaystyle }c<{$}}
\newcolumntype{R}{>{$\displaystyle }r<{$}}

\linespread{1.5}\selectfont

%公式按章节编号
%\makeatletter % `@' now normal "letter"
%\@addtoreset{equation}{section}
%\makeatother  % `@' is restored as "non-letter"
%\renewcommand\theequation{\oldstylenums{\thesection}%
%                   .\oldstylenums{\arabic{equation}}}