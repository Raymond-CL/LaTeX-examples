\documentclass[12pt,a4paper]{article}
\usepackage[margin=3cm]{geometry}
\usepackage{hyperref}
\usepackage{insdljs}
\usepackage{listings}
\usepackage{fontawesome}
\usepackage{textcomp}

\begin{insDLJS}[test]{test}{JavaScript}
function auto()
{
	tts.speaker = tts.getNthSpeakerName(2);
	tts.qText ("Hello, this is a demonstration of a LaTeX document with sound by activating Java script using the insDLJS package.");
	tts.qText ("Please click the speaker icon for code explanation.");
	tts.talk();
}

function script1()
{
   tts.speaker = tts.getNthSpeakerName(2);
   tts.qText ("We first define a Java script function in the front matter.");
   tts.qText ("The text spoken is input in the Q-text bracket.");
   tts.talk();
}

function script2()
{
   var cSpeaker = tts.getNthSpeakerName(2);
   tts.speaker = cSpeaker;
   tts.qText ("Then in the main document, we can include this script via a clickable push button.");
   tts.qText ("When the button is pressed, the function is called, and the script is spoken.");
   tts.talk();
}

function script3()
{
   var cSpeaker = tts.getNthSpeakerName(1);
   tts.speaker = cSpeaker;
   tts.qText ("We see that we can also change the speaker style by using different speaker names.");
   tts.talk();
}

function script4()
{
   var cSpeaker = tts.getNthSpeakerName(0);
   tts.speaker = cSpeaker;
   tts.qText ("We can also include a script that is automatically spoken when the document is opened.");
   tts.talk();
}

function script5()
{
   var cSpeaker = tts.getNthSpeakerName(2);
   tts.speaker = cSpeaker;
   tts.qText ("The package necessary for voice script is ins-D-L-J-S.");
   tts.qText ("The font awesome package is responsible creating the speaker icon.");
   tts.qText ("The hyper-ref package then creates the clickable box around the icon.");
   tts.qText ("The listings package shows the codes in this document.");
   tts.talk();
}
\end{insDLJS}

\OpenAction{/S/JavaScript/JS(auto();)}

\title{insDLJS example}
\date{\today}
\author{Raymond Chen}

\begin{document}

\maketitle

a demonstration of the insDLJS package, please make sure to open this document using Adobe Acrobat\textsuperscript{\textregistered} Reader, and have the speaker volume turned up.
\vspace*{1cm}

\begin{Form}
\PushButton[name=hello, onclick={script1();}, bordercolor={0.650 .790 .94}]{\faVolumeUp} 
\end{Form}
\begin{lstlisting}[basicstyle=\scriptsize]
\begin{insDLJS}[test]{test}{JavaScript}
function script()
{
	tts.speaker = tts.getNthSpeakerName(0);
	tts.qText ("Hello, testing Java script speaker.");
	tts.talk();
}
\end{insDLJS}
\end{lstlisting}
\vspace*{1cm}

\begin{Form}
\PushButton[name=hello, onclick={script2();}, bordercolor={0.650 .790 .94}]{\faVolumeUp} 
\end{Form}
\begin{lstlisting}[basicstyle=\scriptsize]
\begin{document}
...
\begin{Form}
\PushButton[name=anyname, onclick={script();}]{\faVolumeUp}
\end{Form}
...
\end{document}
\end{lstlisting}
\vspace*{1cm}

\begin{Form}
\PushButton[name=hello, onclick={script3();}, bordercolor={0.650 .790 .94}]{\faVolumeUp} 
\end{Form}
\begin{lstlisting}[basicstyle=\scriptsize]
tts.getNthSpeakerName(1)
\end{lstlisting}
\vspace*{1cm}

\begin{Form}
\PushButton[name=hello, onclick={script4();}, bordercolor={0.650 .790 .94}]{\faVolumeUp} 
\end{Form}
\begin{lstlisting}[basicstyle=\scriptsize]
\OpenAction{/S/JavaScript/JS(script();)}
\end{lstlisting}
\vspace*{1cm}

\begin{Form}
\PushButton[name=hello, onclick={script5();}, bordercolor={0.650 .790 .94}]{\faVolumeUp} 
\end{Form}
\begin{lstlisting}[basicstyle=\scriptsize]
\usepackage{insdljs}
\usepackage{fontawesome}
\usepackage{hyperref}
\usepackage{listings}
\end{lstlisting}

\end{document}